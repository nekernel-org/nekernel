\documentclass{article}
\usepackage{graphicx}
\usepackage{hyperref}

\title{CoreProcessScheduler: Technical Documentation}
\author{Amlal El Mahrouss}
\date{\today}

\begin{document}

\maketitle

\section{Abstract}

{The CoreProcessScheduler governs how the scheduling backend and policy of the kernel works, It is the common gateway for schedulers inside NeKernel based systems.}

\section{Overview}

{The CoreProcessScheduler (now referred as CPS) serves as the intermediate foundal between the scheduler backend and kernel.} {It takes care of process life-cycle management, team-based process grouping, and affinity-based CPU based allocation to mention the least.}

\section{The Affinity System}

{Processes are given CPU time affinity hints using an affinity kind type, these hints help the scheduler run or adjust the current process.}

\subsection{Sample Code \#1}

{The following sample is C++ code.} {The smaller the value, the more critical the process.}

\begin{verbatim}
enum class AffinityKind : Int32 {
  kRealTime     = 100,
  kVeryHigh     = 150,
  kHigh         = 200,
  kStandard     = 1000,
  kLowUsage     = 1500,
  kVeryLowUsage = 2000,
};
\end{verbatim}

\section{The Team System}

{The team system holds process metadata for the backend scheduler to run on. It holds methods and fields for backend specific operations.} {One implementation of such team is the UserProcessTeam object inside NeKernel.}

\subsection{Sample Code \#2}

{The following sample is used to hold team metadata.} {This is part of the NeKernel source tree.}

\begin{verbatim}
class UserProcessTeam final {
 public:
  explicit UserProcessTeam();
  ~UserProcessTeam() = default;

  NE_COPY_DEFAULT(UserProcessTeam)

  Array<USER_PROCESS, kSchedProcessLimitPerTeam>& AsArray();
  Ref<USER_PROCESS>&                              AsRef();
  ProcessID&                                      Id() noexcept;

 public:
  USER_PROCESS_ARRAY mProcessList;
  USER_PROCESS_REF   mCurrentProcess;
  ProcessID          mTeamId{0};
  ProcessID          mProcessCur{0};
};

\end{verbatim}

\section{The Process Image System}

{The process image container is a design pattern made to contain process data and metadata, its purpose comes from the lack of mainstream operating systems of such ability to hold metadata.}

{This approach helps separate concerns and give modularity to the system, as the image and process structure are not mixed together.}

\subsection{Sample Code \#3}

{The following sample is a C++ container used to hold process data and metadata.} {This is part of the NeKernel source tree.}

\begin{verbatim}
struct ProcessImage final {
  explicit ProcessImage() = default;

 private:
  friend USER_PROCESS;
  friend KERNEL_TASK;

  friend class UserProcessScheduler;

  ImagePtr fCode;
  ImagePtr fBlob;

 public:
  Bool HasCode() const { return this->fCode != nullptr; }

  Bool HasImage() const { return this->fBlob != nullptr; }

  ErrorOr<ImagePtr> LeakImage() {
    if (this->fCode) {
      return ErrorOr<ImagePtr>{this->fCode};
    }

    return ErrorOr<ImagePtr>{kErrorInvalidData};
  }

  ErrorOr<ImagePtr> LeakBlob() {
    if (this->fBlob) {
      return ErrorOr<ImagePtr>{this->fBlob};
    }

    return ErrorOr<ImagePtr>{kErrorInvalidData};
  }
};

\end{verbatim}

\section{Conclusion}

{The CoreProcessScheduler is a piece of systems design with robust design and useful cases, although useful in desktop/server cases, It may not be suited for every other tasks.}

{And while one scheduler backend (such as the UserProcessScheduler) takes care of user process scheduling and fairness, the CoreProcessScheduler takes care of the foundation for those systems.}

\section{References}

{NeKernel}: \href{https://github.com/nekernel-org/nekernel}{NeKernel}

{VMKernel}: \href{https://snu.systems/specs/vmkernel}{VMKernel}

{CoreProcessScheduler C++ Header}: \href{https://github.com/nekernel-org/nekernel/blob/dev/dev/kernel/KernelKit/CoreProcessScheduler.h}{CoreProcessScheduler}

\end{document}

\documentclass{article}
\usepackage{graphicx} % Required for inserting images

\title{The BinaryMutex Pattern}
\author{Amlal El Mahrouss}
\date{\today}

\begin{document}

\maketitle

\section{Context}

The BinaryMutex is a core component of NeKernel (NeKernel/VMKernel) based systems.

The pattern excludes other acquirers to own the USER\_PROCESS that is currently being hold.

Thus the acquiree is the USER\_PROCESS itself

\section{Use Cases}

\begin{verbatim}
BinaryMutex mux;
mux.Lock(process);

// Say we want to interact with the process itself on this thread, 
we can then make sure that no race condition happens by using:
constexpr auto kSecondsMax = 5;
mux.WaitForProcess(kSecondsMax); 
\end{verbatim}

\section{Implementation}

The source implementation consists of:

\begin{verbatim}
class BinaryMutex final {
 public:
  explicit BinaryMutex() = default;
  ~BinaryMutex()         = default;

 public:
  bool IsLocked() const;
  bool Unlock() noexcept;

 public:
  BOOL WaitForProcess(const UInt32& sec) noexcept;

 public:
  bool Lock(USER_PROCESS* process);
  bool LockOrWait(USER_PROCESS* process, TimerInterface* timer);

 public:
  NE_COPY_DEFAULT(BinaryMutex)

 private:
  USER_PROCESS* fLockingProcess;
};
\end{verbatim}

\section{Conclusion}

This pattern is useful for systems that need to make sure that a process isn't tampered by concurrent usages. Thus its existence in VMKernel and NeKernel.

\end{document}

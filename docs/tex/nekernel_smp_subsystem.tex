\documentclass{article}
\usepackage{graphicx}

\title{NeKernel: The SMP Subsystem}
\author{Amlal El Mahrouss}
\date{\today}

\begin{document}

\maketitle

\section{Abstract}

{NeKernel is a hybrid based operating system kernel written in modern C++ (C++17/C++20). It features a bootloader, kernel, tools, libraries, and frameworks. This document is about the SMP subsystem of NeKernel}

\section{Design Overview}

ne\_kernel is designed with SMP by default. Although it may fallback under classic preemptive round-robin scheduling when unavailable - NeKernel runs best on a SMP based machine. The subsystem goes from the HardwareThreadScheduler to the Hardware Abstraction Layer's Application Processors API.

\subsection{HardwareThreadScheduler (HTS)}

HardwareThreadScheduler's main purpose is to make cores all busy with a StackFrame. That StackFrame contains program information (instruction, stack pointers) each task is fairly assigned to a core to then be run by the Application Processors API.

\subsection{Application Processors (AP)}

Application Processors (now referred as AP) is the API taking care of multi-core scheduling, very platform dependent (thus its presence on the HAL) it is designed to run tasks passed from the HardwareThreadScheduler system.

\end{document}

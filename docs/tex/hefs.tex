\documentclass{article}
\usepackage[a4paper, margin=1in]{geometry}
\usepackage{amsmath, amssymb}
\usepackage{listings}
\usepackage{xcolor}
\usepackage{graphicx}
\usepackage{enumitem}
\usepackage{caption}
\usepackage{longtable}

\definecolor{lightgray}{gray}{0.95}

\lstdefinestyle{cstyle}{
  language=C++,
  backgroundcolor=\color{lightgray},
  basicstyle=\ttfamily\small,
  keywordstyle=\color{blue},
  commentstyle=\color{green!60!black},
  stringstyle=\color{orange},
  numbers=left,
  numberstyle=\tiny,
  breaklines=true,
  frame=single,
  showstringspaces=false
}

\title{HeFS (High-Throughput Extended File System) Specification}
\author{Amlal El Mahrouss}
\date{2024-2025}

\begin{document}

\maketitle

\section{Overview}
HeFS is a high-throughput filesystem designed for NeKernel. It implements a robust directory structure based on red-black trees, uses slice-linked blocks for file storage, and includes CRC32-based integrity checks. Designed for desktop, server, and embedded contexts, it prioritizes performance, corruption recovery, and extensibility.

\section{Boot Node Structure}
\begin{longtable}{|l|l|p{8cm}|}
\hline
\textbf{Field} & \textbf{Type} & \textbf{Description} \\
\hline
\verb|fMagic| & \verb|char[8]| & Filesystem magic ("  HeFS") \\
\verb|fVolName| & \verb|Utf8Char[128]| & Volume name \\
\verb|fVersion| & \verb|UInt32| & Filesystem version (e.g., 0x0102) \\
\verb|fBadSectors| & \verb|UInt64| & Number of bad sectors detected \\
\verb|fSectorCount| & \verb|UInt64| & Total sector count \\
\verb|fSectorSize| & \verb|UInt64| & Size of each sector \\
\verb|fChecksum| & \verb|UInt32| & CRC32 checksum of the boot node \\
\verb|fDiskKind| & \verb|UInt8| & Type of drive (e.g., HDD, SSD, USB) \\
\verb|fEncoding| & \verb|UInt8| & Encoding mode (UTF-8, UTF-16, etc.) \\
\verb|fStartIND| & \verb|UInt64| & Starting LBA of inode directory region \\
\verb|fEndIND| & \verb|UInt64| & Ending LBA of inode directory region \\
\verb|fINDCount| & \verb|UInt64| & Number of directory nodes allocated \\
\verb|fDiskSize| & \verb|UInt64| & Logical size of the disk \\
\verb|fDiskStatus| & \verb|UInt16| & Status of the disk (e.g., unlocked, locked) \\
\verb|fDiskFlags| & \verb|UInt16| & Disk flags (e.g., read-only) \\
\verb|fVID| & \verb|UInt16| & Virtual ID (for EPM integration) \\
\hline
\end{longtable}

\section{File Types and Flags}
\subsection*{File Kinds}
\begin{itemize}[label=--]
    \item \verb|0x00| --- Regular File
    \item \verb|0x01| --- Directory
    \item \verb|0x02| --- Block Device
    \item \verb|0x03| --- Character Device
    \item \verb|0x04| --- FIFO
    \item \verb|0x05| --- Socket
    \item \verb|0x06| --- Symbolic Link
    \item \verb|0x07| --- Unknown
\end{itemize}

\subsection*{Drive Types}
\begin{itemize}[label=--]
    \item \verb|0xC0| --- Hard Drive
    \item \verb|0xC1| --- Solid State Drive
    \item \verb|0x0C| --- Optical Drive
    \item \verb|0xCC| --- USB Mass Storage
    \item \verb|0xC4| --- SCSI/SAS Drive
    \item \verb|0xC6| --- Flash Drive
\end{itemize}

\section{Index Node Structure}
The `HEFS\_INDEX\_NODE` represents a file and is constrained to 512 bytes to match hardware sector boundaries. It uses a fixed set of block pointers (slices) and CRC32 checks for data integrity. Only the local file name is stored in `fName`.

\begin{lstlisting}[style=cstyle, caption={HEFS\_INDEX\_NODE (Fits 512B)}]
struct HEFS_INDEX_NODE {
    UInt64   fHashPath;        // Local file name, hashed.
    UInt32   fFlags;
    UInt16   fKind;
    UInt32   fSize;
    UInt32   fChecksum;

    Boolean  fSymLink;
    ATime    fCreated, fAccessed, fModified, fDeleted;
    UInt32   fUID, fGID;
    UInt32   fMode;

    UInt64   fSlices[16];        // Data block slices (start-only)
    Char     fPad[317];          // Padding to reach 512B
};
\end{lstlisting}

\section{Directory Node Structure}
Directories form a red-black tree. Each node (IND) can hold up to 16 index node references and links to its parent, siblings, and children via LBAs.

\begin{lstlisting}[style=cstyle, caption={HEFS\_INDEX\_NODE\_DIRECTORY}]
struct HEFS_INDEX_NODE_DIRECTORY {
    UInt64   fHashPath;        // Directory path as hash
    UInt32   fFlags;
    UInt16   fKind;
    UInt32   fEntryCount;
    UInt32   fChecksum;

    ATime    fCreated, fAccessed, fModified, fDeleted;
    UInt32   fUID, fGID;
    UInt32   fMode;

    UInt64   fINSlices[16];    // Inode LBA references

    UInt8    fColor;            // Red/Black tree color
    Lba      fNext, fPrev, fChild, fParent;

    Char     fPad[285];
};
\end{lstlisting}

\section{Design Characteristics}

\begin{itemize}
    \item Red-black tree traversal for directory balancing
    \item One-sector (512B) inode design for efficient I/O
    \item Slice-linked file storage (fixed 16 slots)
    \item CRC32 for boot node, inode, and directory integrity
    \item Preallocated directory inodes to avoid runtime fragmentation
\end{itemize}

\section{Minimum Requirements}

\begin{itemize}
    \item Minimum disk size: 16MB
    \item Recommended size: 8GB or more
    \item Supported media: HDD, SSD, USB, SCSI, Flash, Optical
\end{itemize}

\section{Future Work}
\begin{itemize}
    \item Journaling layer for recovery
    \item Extended access control and ACL support
    \item Logical Volume Management (LVM)
    \item Dual boot node layout for redundancy
    \item Self-healing and online fsck tools
\end{itemize}

\end{document}

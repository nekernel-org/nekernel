\documentclass{article}
\usepackage[a4paper, margin=1in]{geometry}
\usepackage{amsmath, amssymb}
\usepackage{listings}
\usepackage{xcolor}
\usepackage{graphicx}
\usepackage{enumitem}
\usepackage{caption}
\usepackage{longtable}

\definecolor{lightgray}{gray}{0.95}

\lstdefinestyle{cstyle}{
  language=C++,
  backgroundcolor=\color{lightgray},
  basicstyle=\ttfamily\small,
  keywordstyle=\color{blue},
  commentstyle=\color{green!60!black},
  stringstyle=\color{orange},
  numbers=left,
  numberstyle=\tiny,
  breaklines=true,
  frame=single,
  showstringspaces=false
}

\title{HeFS (Hierarchical Embedded File System) Specification}
\author{Amlal El Mahrouss}
\date{2024–2025}

\begin{document}

\maketitle

\section{Overview}
HeFS is a custom filesystem developed as part of the NeKernel project. It offers a journaling-like inode tree structure using red-black tree-inspired navigation for directories. Designed for robust use in desktop and server workloads, it supports various encoding modes and drive types.

\section{Boot Node Structure}
\begin{longtable}{|l|l|p{8cm}|}
\hline
\textbf{Field} & \textbf{Type} & \textbf{Description} \\
\hline
\verb|fMagic| & \verb|char[8]| & Filesystem magic ("  HeFS") \\
\verb|fVolName| & \verb|Utf16Char[128]| & Volume name \\
\verb|fVersion| & \verb|UInt32| & Filesystem version (e.g., 0x0100) \\
\verb|fSectorCount| & \verb|UInt64| & Total sector count \\
\verb|fSectorSize| & \verb|UInt64| & Size of each sector \\
\verb|fDriveKind| & \verb|UInt8| & Type of drive (e.g., HDD, SSD, USB) \\
\verb|fEncoding| & \verb|UInt8| & Encoding mode (UTF-8, UTF-16, etc.) \\
\verb|fStartIND| & \verb|UInt64| & Starting LBA of inode tree \\
\verb|fEndIND| & \verb|UInt64| & Ending LBA of inode tree \\
\verb|fChecksum| & \verb|UInt32| & CRC32 of boot node \\
\verb|fDiskSize| & \verb|UInt64| & Logical size of the disk \\
\hline
\end{longtable}

\section{File Types and Flags}
\subsection*{File Kinds}
\begin{itemize}[label=--]
    \item \verb|0x00| — Regular File
    \item \verb|0x01| — Directory
    \item \verb|0x02| — Block Device
    \item \verb|0x03| — Character Device
    \item \verb|0x04| — FIFO
    \item \verb|0x05| — Socket
    \item \verb|0x06| — Symbolic Link
\end{itemize}

\subsection*{Drive Types}
\begin{itemize}[label=--]
    \item \verb|0xC0| — Hard Drive
    \item \verb|0xC1| — SSD
    \item \verb|0xCC| — Mass Storage Device (USB)
\end{itemize}

\section{Index Node Structure}
The inode tree allows for fast access and recovery via dual start/end block mappings and timestamped metadata.

\begin{lstlisting}[style=cstyle, caption={HEFS\_INDEX\_NODE structure}]
struct HEFS_INDEX_NODE {
    Utf16Char fName[256];
    UInt32    fFlags;
    UInt16    fKind;
    UInt32    fSize;
    UInt32    fChecksum, fRecoverChecksum, fBlockChecksum, fLinkChecksum;
    ATime     fCreated, fAccessed, fModified, fDeleted;
    UInt32    fUID, fGID;
    UInt32    fMode;
    UInt64    fBlockLinkStart[16], fBlockLinkEnd[16];
    UInt64    fBlockStart[16], fBlockEnd[16];
    UInt64    fBlockRecoveryStart[16], fBlockRecoveryEnd[16];
};
\end{lstlisting}

\section{Directory Node Structure}
Each directory is a red-black tree node with child and sibling pointers.

\begin{lstlisting}[style=cstyle, caption={HEFS\_INDEX\_NODE\_DIRECTORY structure}]
struct HEFS_INDEX_NODE_DIRECTORY {
    Utf16Char fName[256];
    UInt32 fFlags;
    UInt16 fKind;
    UInt32 fSize;
    UInt32 fChecksum, fIndexNodeChecksum;
    ATime  fCreated, fAccessed, fModified, fDeleted;
    UInt32 fUID, fGID;
    UInt32 fMode;
    UInt64 fIndexNodeStart[16], fIndexNodeEnd[16];
    UInt8  fColor;
    Lba    fNext, fPrev, fChild, fParent;
};
\end{lstlisting}

\section{Future Work}
Planned extensions include:
\begin{itemize}
    \item Journaling recovery logic
    \item Advanced permission flags
    \item Logical volume management support
    \item Cross-filesystem linking (EPM integration)
\end{itemize}

\end{document}
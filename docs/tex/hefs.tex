\documentclass{article}
\usepackage[a4paper, margin=1in]{geometry}
\usepackage{amsmath, amssymb}
\usepackage{listings}
\usepackage{xcolor}
\usepackage{graphicx}
\usepackage{enumitem}
\usepackage{caption}
\usepackage{longtable}

\definecolor{lightgray}{gray}{0.95}

\lstdefinestyle{cstyle}{
  language=C++,
  backgroundcolor=\color{lightgray},
  basicstyle=\ttfamily\small,
  keywordstyle=\color{blue},
  commentstyle=\color{green!60!black},
  stringstyle=\color{orange},
  numbers=left,
  numberstyle=\tiny,
  breaklines=true,
  frame=single,
  showstringspaces=false
}

\title{HeFS (High-Throughput Extended File System) Specification}
\author{Amlal El Mahrouss}
\date{2024-2025}

\begin{document}

\maketitle

\section{Overview}
HeFS is a custom filesystem developed as part of the NeKernel project. It offers a journaling-like inode tree structure using red-black tree-inspired navigation for directories. Designed for robust use in desktop, server, and embedded environments, it supports various encoding modes and drive types.

\section{Boot Node Structure}
\begin{longtable}{|l|l|p{8cm}|}
\hline
\textbf{Field} & \textbf{Type} & \textbf{Description} \\
\hline
\verb|fMagic| & \verb|char[8]| & Filesystem magic ("  HeFS") \\
\verb|fVolName| & \verb|Utf16Char[128]| & Volume name \\
\verb|fVersion| & \verb|UInt32| & Filesystem version (e.g., 0x0101) \\
\verb|fBadSectors| & \verb|UInt64| & Number of bad sectors detected \\
\verb|fSectorCount| & \verb|UInt64| & Total sector count \\
\verb|fSectorSize| & \verb|UInt64| & Size of each sector \\
\verb|fChecksum| & \verb|UInt32| & CRC32 checksum of the boot node \\
\verb|fDiskKind| & \verb|UInt8| & Type of drive (e.g., HDD, SSD, USB) \\
\verb|fEncoding| & \verb|UInt8| & Encoding mode (UTF-8, UTF-16, etc.) \\
\verb|fStartIND| & \verb|UInt64| & Starting LBA of inode tree \\
\verb|fEndIND| & \verb|UInt64| & Ending LBA of inode tree \\
\verb|fINDCount| & \verb|UInt64| & Number of directory nodes allocated \\
\verb|fDiskSize| & \verb|UInt64| & Logical size of the disk \\
\verb|fDiskStatus| & \verb|UInt16| & Status of the disk (e.g., unlocked, locked) \\
\verb|fDiskFlags| & \verb|UInt16| & Disk flags (e.g., read-only) \\
\verb|fVID| & \verb|UInt16| & Virtual ID (EPM integration) \\
\hline
\end{longtable}

\section{File Types and Flags}
\subsection*{File Kinds}
\begin{itemize}[label=--]
    \item \verb|0x00| --- Regular File
    \item \verb|0x01| --- Directory
    \item \verb|0x02| --- Block Device
    \item \verb|0x03| --- Character Device
    \item \verb|0x04| --- FIFO
    \item \verb|0x05| --- Socket
    \item \verb|0x06| --- Symbolic Link
    \item \verb|0x07| --- Unknown
\end{itemize}

\subsection*{Drive Types}
\begin{itemize}[label=--]
    \item \verb|0xC0| --- Hard Drive
    \item \verb|0xC1| --- Solid State Drive
    \item \verb|0x0C| --- Optical Drive
    \item \verb|0xCC| --- USB Mass Storage
    \item \verb|0xC4| --- SCSI/SAS Drive
    \item \verb|0xC6| --- Flash Drive
\end{itemize}

\section{Index Node Structure}
Files are stored through block links, offering native recovery fields and MIME type support.

\begin{lstlisting}[style=cstyle, caption={HEFS\_INDEX\_NODE structure}]
struct HEFS_INDEX_NODE {
    Utf16Char fName[256];
    UInt32    fFlags;
    UInt16    fKind;
    UInt32    fSize;
    UInt32    fChecksum, fRecoverChecksum, fBlockChecksum, fLinkChecksum;
    Utf16Char fMime[256];
    Boolean   fSymLink;
    ATime     fCreated, fAccessed, fModified, fDeleted;
    UInt32    fUID, fGID;
    UInt32    fMode;
    UInt64    fBlockLinkStart[16], fBlockLinkEnd[16];
    UInt64    fBlockStart[16], fBlockEnd[16];
    UInt64    fBlockRecoveryStart[16], fBlockRecoveryEnd[16];
};
\end{lstlisting}

\section{Directory Node Structure}
Directories are organized into a red-black tree for efficient balancing.

\begin{lstlisting}[style=cstyle, caption={HEFS\_INDEX\_NODE\_DIRECTORY structure}]
struct HEFS_INDEX_NODE_DIRECTORY {
    Utf16Char fName[256];
    UInt32    fFlags;
    UInt16    fKind;
    UInt32    fEntryCount;
    UInt32    fChecksum, fIndexNodeChecksum;
    Utf16Char fDim[256];
    ATime     fCreated, fAccessed, fModified, fDeleted;
    UInt32    fUID, fGID;
    UInt32    fMode;
    UInt64    fIndexNodeStart[16], fIndexNodeEnd[16];
    UInt8     fColor;
    Lba       fNext, fPrev, fChild, fParent;
};
\end{lstlisting}

\section{Filesystem Design}

HeFS is designed with the following objectives:
\begin{itemize}
    \item Red-black tree navigation for efficient directory balancing
    \item Journaling fields for block-level recovery
    \item Multi-encoding support: UTF-8, UTF-16, UTF-32
    \item Advanced MIME type support
    \item Redundant fields (checksums, recovery inodes) for crash resistance
    \item Extensible for future LVM (Logical Volume Management) and network filesystem support
\end{itemize}

\section{Minimum Requirements}

HeFS expects a minimum disk size of 16MB. Optimal performance is aimed to be achieved with 8GB or more.
Supports: HDDs, SSDs, USB mass storage, SCSI/SAS, optical drives.

\section{Future Work}
Planned enhancements include:
\begin{itemize}
    \item Full journaling implementation (recovery on crash)
    \item Advanced ACLs (Access Control Lists) and permissions
    \item Logical Volume Management (LVM) integration
    \item Backup Superblock and dual-boot sectors
    \item Online filesystem checking and self-healing algorithms
\end{itemize}

\end{document}
